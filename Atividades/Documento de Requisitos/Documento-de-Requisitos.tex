%% MODELO DE LATEX PARA TRABALHOS ACADÊMICOS
%% INSTRUÇÕES GERAIS:
%%    1. TODO O TEXTO NA FRENTE DO SIMBOLO '%' É COMENTÁRIO, ISTO É, ELE NÃO FAZ DIFERENÇA NO RESULTADO FINAL 
%%    2. NESTE MODELO, VOCÊS SÓ PRECISAM EDITAR DAS LINHAS 114 A 132 (INFORMAÇÕES DE CAPA) E DAS LINHAS 188 EM DIANTE (CORPO DO TRABALHO). O RESTO SÃO CONFIGURAÇÕES DE FORMATAÇÃO QUE PROVAVELMENTE NÃO SERÁ PRECISO MODIFICAR.
%%    3. MAIS INSTRUÇÕES DETALHADAS PODERÃO SER ENCONTRADAS NA PÁGINA profhelioh.wordpress.com. DÚVIDAS: heliohenrique@ufpr.br OU heliohenrique3@gmail.com

% INFORMAÇÕES DA FONTE:
%% abtex2-modelo-relatorio-tecnico.tex, v-1.7.1 laurocesar
%% Copyright 2012-2013 by abnTeX2 group at http://abntex2.googlecode.com/ 
%%
%% This work may be distributed and/or modified under the
%% conditions of the LaTeX Project Public License, either version 1.3
%% of this license or (at your option) any later version.
%% The latest version of this license is in
%%   http://www.latex-project.org/lppl.txt
%% and version 1.3 or later is part of all distributions of LaTeX
%% version 2005/12/01 or later.
%%
%% This work has the LPPL maintenance status `maintained'.
%% 
%% The Current Maintainer of this work is the abnTeX2 team, led
%% by Lauro César Araujo. Further information are available on 
%% http://abntex2.googlecode.com/
%%
%% This work consists of the files abntex2-modelo-relatorio-tecnico.tex,
%% abntex2-modelo-include-comandos and abntex2-modelo-references.bib
%%
% ------------------------------------------------------------------------
% ------------------------------------------------------------------------
% abnTeX2: Modelo de Relatório Técnico/Acadêmico em conformidade com 
% ABNT NBR 10719:2011 Informação e documentação - Relatório técnico e/ou
% científico - Apresentação
% ------------------------------------------------------------------------ 
% ------------------------------------------------------------------------

\documentclass[
	% -- opções da classe memoir --
	12pt,				% tamanho da fonte
	% openright,			% capítulos começam em pág ímpar (insere página vazia caso preciso)
    oneside,			% para impressão somente frente. Oposto a twoside (frente e verso)
	a4paper,			% tamanho do papel. 
	% -- opções da classe abntex2 --
	chapter=TITLE,		% títulos de capítulos convertidos em letras maiúsculas
	%section=TITLE,		% títulos de seções convertidos em letras maiúsculas
	%subsection=TITLE,	% títulos de subseções convertidos em letras maiúsculas
	%subsubsection=TITLE,% títulos de subsubseções convertidos em letras maiúsculas
	% -- opções do pacote babel --
	english,			% idioma adicional para hifenização
	french,				% idioma adicional para hifenização
	spanish,			% idioma adicional para hifenização
	brazil,				% o último idioma é o principal do documento
	]{abntex2}


% ---
% PACOTES
% ---

% ---
% Pacotes fundamentais 
% ---
\usepackage{cmap}				% Mapear caracteres especiais no PDF
\usepackage{lmodern}			% Usa a fonte Latin Modern
\usepackage[T1]{fontenc}		% Selecao de codigos de fonte.
\usepackage[utf8]{inputenc}		% Codificacao do documento (conversão automática dos acentos)
\usepackage{indentfirst}		% Indenta o primeiro parágrafo de cada seção.
\usepackage{color}				% Controle das cores
\usepackage{graphicx}			% Inclusão de gráficos
% ---

% ---
% Pacotes adicionais, usados no anexo do modelo de folha de identificação
% ---
\usepackage{multicol}
\usepackage{multirow}
% ---
\usepackage{array}
% ---
% Pacotes adicionais, usados apenas no âmbito do Modelo Canônico do abnteX2
% ---
\usepackage{lipsum}				% para geração de dummy text
% ---

% ---
% Pacotes de citações
% ---
\usepackage[brazilian,hyperpageref]{backref}	 % Paginas com as citações na bibl
\usepackage[alf]{abntex2cite}	% Citações padrão ABNT

% --- 
% CONFIGURAÇÕES DE PACOTES
% --- 

% ---
% Configurações do pacote backref
% Usado sem a opção hyperpageref de backref
\renewcommand{\backrefpagesname}{Citado na(s) página(s):~}
% Texto padrão antes do número das páginas
\renewcommand{\backref}{}
% Define os textos da citação
\renewcommand*{\backrefalt}[4]{
	\ifcase #1 %
		Nenhuma citação no texto.%
	\or
		Citado na página #2.%
	\else
		Citado #1 vezes nas páginas #2.%
	\fi}%
% ---

% ---
% Informações de dados para CAPA e FOLHA DE ROSTO
% ---
\titulo{Documento de Especificação de Requisitos de Software: Médida Protetiva Online}
\autor{Rafael Gonçalves de Oliveira Viana}
\local{Brasil}
\data{14 de junho de 2017}
\instituicao{%
  Universidade Federal do Mato Grosso do Sul
  \par
  Câmpus Coxim
  \par
  Sistema de Informação}
\tipotrabalho{Documento de Especificação de Requisitos de Software}
% O preambulo deve conter o tipo do trabalho, o objetivo, 
% o nome da instituição e a área de concentração 
\preambulo{Levantamento de requisitos, referente ao trabalho I, realizado na matéria de Engenharia Web I, ministrada pelo M. Prof. Marcel Seiji Kay}
% ---

% ---
% Configurações de aparência do PDF final

% alterando o aspecto da cor azul
\definecolor{blue}{RGB}{41,5,195}

% informações do PDF
\makeatletter
\hypersetup{
     	%pagebackref=true,
		pdftitle={\@title}, 
		pdfauthor={\@author},
    	pdfsubject={\imprimirpreambulo},
	    pdfcreator={LaTeX with abnTeX2},
		pdfkeywords={abnt}{latex}{abntex}{abntex2}{relatório técnico}, 
		colorlinks=true,       		% false: boxed links; true: colored links
    	linkcolor=blue,          	% color of internal links
    	citecolor=blue,        		% color of links to bibliography
    	filecolor=magenta,      		% color of file links
		urlcolor=blue,
		bookmarksdepth=4
}
\makeatother
% --- 

% --- 
% Espaçamentos entre linhas e parágrafos 
% --- 

% O tamanho do parágrafo é dado por:
\setlength{\parindent}{1.3cm}

% Controle do espaçamento entre um parágrafo e outro:
\setlength{\parskip}{0.2cm}  % tente também \onelineskip

% ---
% compila o indice
% ---
\makeindex
% ---

% ----
% Início do documento
% ----
\begin{document}

% Retira espaço extra obsoleto entre as frases.
\frenchspacing 

% ----------------------------------------------------------
% ELEMENTOS PRÉ-TEXTUAIS
% ----------------------------------------------------------

\imprimircapa

\imprimirfolhaderosto*

\tableofcontents

% ----------------------------------------------------------
% ELEMENTOS TEXTUAIS  (necessário para incluir número nas páginas)
% ----------------------------------------------------------
\textual


% ----------------------------------------------------------
% Introdução
% ----------------------------------------------------------

\chapter{Documento de Requisitos} %% NOVO CAPÍTULO (REPARE QUE ELE AUTOMATICAMENTE JÁ COLOCA O NÚMERO DO CAPÍTULO E JÁ ADICIONA NO SUMÁRIO)
	\section{Descrição do Sistema}
		O sistema para Policia Militar consiste em desenvolver um gerenciador de medidas protetivas, expedida pela Justiça. Cada medida tem uma ou mais vitimas e um ou mais Reus, contendo data de emissão, objservação e um valor em dias que representa quantos dias esta em vigor essa medida. Com a medida já em mão deve-se cadastrala colocando esses dados para possível consulta e impressão de uma versão previamente digitalizada.
	
		O sistema deve ainda emitir diversos tipos de relatórios e consultas, possibilitando um melhor gerenciamento do policiamento ostencivo economisando recursos para melhores aplicações.
		
	\section{Objetivos do documento}
	
		Este documento tem o objetivo de definir os requisitos do sistema Medida Protetiva. Nele estão contemplados os requisitos funcionais e não-funcionais, casos de uso e regras de negócio do sistema.
	
	\section{Escopo do documento}
	
		Este documento define os requisitos funcionais e não funcionais casos de uso e regras de negócios correlatos para o sistema acima citado. Para elucidação de termos específicos utilizados neste documento, faz-se necessária referência ao Glossário.

	\section {Requisitos Funcionais}\label{RF}
	
		\subsection{Login (Usuário/Gerenciador)}
			\begin{enumerate}
				\item O sistema deve possuir 2 niveis de usuários com diferentes privilégios no sistma sendo eles: gerenciador e usuário. 
				 \item O gerenciador abastecer o sistema podendo consultar e imprimir relatórios.
				 \item O usuário apenas consulta e imprimi relatórios.
			
			\end{enumerate}
		 
		\subsection{Lançamentos das medidas protetivas} \label{RFN1}
	    	\begin{enumerate} 
				\item O sistema deve permitir que apenas o usuário gerenciador realize a inclusão, alteração e remoção de medidas protetivas scaneadas em PDF e anexadas, juntamente do cadastro individual contendo os seguintes atributos: nome, endereço, cidade onde mora, estado, país, telefone, documento de identificação (RG ou CPF para brasileiros e passaporte para estrangeiros), data de nascimento e nome dos pais(“se constar na medida”) de ambas as vitimas e acusados, que constam na médida protetiva.\label{teste}
			
		   	\end{enumerate}
   	
		\subsection{Impressão de relatórios e consultas}
			\begin{enumerate}
				\item O sistema deve criar relatórios e gráficos de bairro, cidades, vitimas e acusados com maiores e menores índices de medidas protetivas, cada relatório deve ser individual para cada uma das opções citadas acima, os dois niveis de usuários poderam acessar essa opção.
				\item O sistema deve permitir a consulta de uma medida protetiva online, pelo nome da vítima, nome do acusado ou pelo cpf dos mesmos, tento acesso ao documento original scaneado e armazenado online.
			\end{enumerate}
	

	\section{Requisitos não funcionais}
	
		\subsection{Confiabilidade}
			\begin{enumerate}
				\item O sistema deve ter capacidade para recuperar os dados perdidos da última operação que realizou em caso de falha.
				\item O sistema deve fornecer ferramentas para a realização de backups dos arquivos do sistema.
				\item O sistema deve possuir senhas de acesso e identificação para diferentes tipos de usuários: usuário e gerenciador.
			\end{enumerate}

	
		\subsection{Eficiência}
			\begin{enumerate}
				\item O sistema deve responder a consultas on-line em menos de 5 segundos.
				\item O sistema deve iniciar a impressão de relatórios solicitados dentro de no máximo 20 segundos após sua requisição.
			\end{enumerate}
		\subsection{Portabilidade}
			\begin{enumerate}
				\item O sistema deve ser web e adaptavel para dispositivos moveis.
			\end{enumerate}
		\subsection{Disponibilidade}
			\begin{enumerate}
				\item O sistema deve estar 24 horas nos 7 dias da semenana online.
			\end{enumerate}
%\chapter{Caso de Uso}		
	\section{Lista de Caso de Uso}
		Alguns dos casos de uso necessários para a implementação do sistema estão listados nesta seção:
		\begin{enumerate}
			\item Cadastro de Atores no sistema.
			\item Login de Atores no Sistema.
			\item Cadastro de Médida Protetiva no sistema.
			\item Consulta de Médida Protetiva no sistema.
			\item Editar Médida Protetiva do sistema.
			\item Excluir Médida Protetiva do sistema.
		\end{enumerate}
	\section{Mapeamento de caso de uso}
		 Nesta seção os requisitos funcionais (RF) serão mapeados em casos de uso. Os casos de uso que serão implementados, estarão descritos detalhadamente na seção \ref{RF}.

		\begin{center}
			\begin{tabular}{ |c|c|c|c| } 
				\hline
				Requisitos Funcionais
				RFNN  & Casos de uso envolvidos
				UCNN \\
				\hline
				~RFN \ref{RFN1} & cell2 \\ 
				\hline
				~RFN \ref{RFN1} & cell2 \\ 
				\hline
			
			\end{tabular}
		\end{center}
	

		
	\section{Lista de Regras de negócio}
		\begin{enumerate}
			\item O sistema deve ter capacidade para recuperar os dados perdidos da última operação que realizou em caso de falha.
			\item O sistema deve fornecer facilidades para a realização de backups dos arquivos do sistema.
			\item O sistema deve possuir senhas de acesso e identificação para diferentes tipos de usuários: usuario e gerenciador
		\end{enumerate}
	
	
	\section{Mapeamento de regras de negócio}	
	



% ----------------------------------------------------------
% ELEMENTOS PÓS-TEXTUAIS
% ----------------------------------------------------------
\postextual


% ----------------------------------------------------------
% Referências bibliográficas
% ----------------------------------------------------------
%\bibliography{abntex2-modelo-references} %% REFERENCIA AO ARQUIVO abntex2-modelo-references.bib

% ----------------------------------------------------------
% Glossário
% ----------------------------------------------------------
%
% Consulte o manual da classe abntex2 para orientações sobre o glossário.
%
%\glossary

% ----------------------------------------------------------
% Apêndices
% ----------------------------------------------------------

% ---
% Inicia os apêndices
% ---
\begin{apendicesenv}

% Imprime uma página indicando o início dos apêndices
\partapendices

% ----------------------------------------------------------
%\chapter{Quisque libero justo}
% ----------------------------------------------------------

%\lipsum[50]


\end{apendicesenv}
% ---


% ----------------------------------------------------------
% Anexos
% ----------------------------------------------------------

%---------------------------------------------------------------------
% INDICE REMISSIVO
%---------------------------------------------------------------------

\printindex

%---------------------------------------------------------------------
% Formulário de Identificação (opcional)
%---------------------------------------------------------------------
\chapter*[Médida Protetva]{Médida Protetva}
\addcontentsline{toc}{chapter}{Exemplo Formato Medida Protetva}
\label{formulado-identificacao}

Exemplo de Medid

\bigskip

\begin{tabular}{|p{9cm}|p{5cm}|} %% EXEMPLO DE TABELA MAIS COMPLEXA
\hline
\multicolumn{2}{|c|}{\textbf{\large Médida Protetiva}}\\
\hline
\multirow{4}{10cm}[24pt]{Título e subtítulo}& Classificação de segurança\\
                   & \\
                   \cline{2-2}
                   & No.\\
                   & \\
				
\hline
Tipo de relatório & Data\\
\hline
Título do projeto/programa/plano & No.\\
\hline
\multicolumn{2}{|l|}{Autor(es)} \\
\hline
\multicolumn{2}{|l|}{Instituição executora e endereço completo} \\
\hline
\multicolumn{2}{|l|}{Instituição patrocinadora e endereço completo} \\
\hline
\multicolumn{2}{|l|}{Resumo}\\[3cm]
\hline
\multicolumn{2}{|l|}{Palavras-chave/descritores}\\
\hline
\multicolumn{2}{|l|}{
Edição \hfill No. de páginas \hfill No. do volume \hfill Nº de classificação \phantom{XXXX}} \\
\hline
\multicolumn{2}{|l|}{
ISSN \hfill \hfill Tiragem \hfill Preço \phantom{XXXXXXXX}} \\
\hline
\multicolumn{2}{|l|}{Distribuidor} \\
\hline
\multicolumn{2}{|l|}{Observações/notas}\\[3cm]
\hline
\end{tabular}

\end{document}
