%% MODELO DE LATEX PARA TRABALHOS ACADÊMICOS
%% INSTRUÇÕES GERAIS:
%%    1. TODO O TEXTO NA FRENTE DO SIMBOLO '%' É COMENTÁRIO, ISTO É, ELE NÃO FAZ DIFERENÇA NO RESULTADO FINAL 
%%    2. NESTE MODELO, VOCÊS SÓ PRECISAM EDITAR DAS LINHAS 114 A 132 (INFORMAÇÕES DE CAPA) E DAS LINHAS 188 EM DIANTE (CORPO DO TRABALHO). O RESTO SÃO CONFIGURAÇÕES DE FORMATAÇÃO QUE PROVAVELMENTE NÃO SERÁ PRECISO MODIFICAR.
%%    3. MAIS INSTRUÇÕES DETALHADAS PODERÃO SER ENCONTRADAS NA PÁGINA profhelioh.wordpress.com. DÚVIDAS: heliohenrique@ufpr.br OU heliohenrique3@gmail.com

% INFORMAÇÕES DA FONTE:
%% abtex2-modelo-relatorio-tecnico.tex, v-1.7.1 laurocesar
%% Copyright 2012-2013 by abnTeX2 group at http://abntex2.googlecode.com/ 
%%
%% This work may be distributed and/or modified under the
%% conditions of the LaTeX Project Public License, either version 1.3
%% of this license or (at your option) any later version.
%% The latest version of this license is in
%%   http://www.latex-project.org/lppl.txt
%% and version 1.3 or later is part of all distributions of LaTeX
%% version 2005/12/01 or later.
%%
%% This work has the LPPL maintenance status `maintained'.
%% 
%% The Current Maintainer of this work is the abnTeX2 team, led
%% by Lauro César Araujo. Further information are available on 
%% http://abntex2.googlecode.com/
%%
%% This work consists of the files abntex2-modelo-relatorio-tecnico.tex,
%% abntex2-modelo-include-comandos and abntex2-modelo-references.bib
%%
% ------------------------------------------------------------------------
% ------------------------------------------------------------------------
% abnTeX2: Modelo de Relatório Técnico/Acadêmico em conformidade com 
% ABNT NBR 10719:2011 Informação e documentação - Relatório técnico e/ou
% científico - Apresentação
% ------------------------------------------------------------------------ 
% ------------------------------------------------------------------------

\documentclass[
	% -- opções da classe memoir --
	12pt,				% tamanho da fonte
	% openright,			% capítulos começam em pág ímpar (insere página vazia caso preciso)
    oneside,			% para impressão somente frente. Oposto a twoside (frente e verso)
	a4paper,			% tamanho do papel. 
	% -- opções da classe abntex2 --
	%chapter=TITLE,		% títulos de capítulos convertidos em letras maiúsculas
	%section=TITLE,		% títulos de seções convertidos em letras maiúsculas
	%subsection=TITLE,	% títulos de subseções convertidos em letras maiúsculas
	%subsubsection=TITLE,% títulos de subsubseções convertidos em letras maiúsculas
	% -- opções do pacote babel --
	english,			% idioma adicional para hifenização
	french,				% idioma adicional para hifenização
	spanish,			% idioma adicional para hifenização
	brazil,				% o último idioma é o principal do documento
	]{abntex2}


% ---
% PACOTES
% ---

% ---
% Pacotes fundamentais 
% ---
\usepackage{cmap}				% Mapear caracteres especiais no PDF
\usepackage{lmodern}			% Usa a fonte Latin Modern
\usepackage[T1]{fontenc}		% Selecao de codigos de fonte.
\usepackage[utf8]{inputenc}		% Codificacao do documento (conversão automática dos acentos)
\usepackage{indentfirst}		% Indenta o primeiro parágrafo de cada seção.
\usepackage{color}				% Controle das cores
\usepackage{graphicx}			% Inclusão de gráficos
% ---
\usepackage{float}

% ---
% Pacotes adicionais, usados no anexo do modelo de folha de identificação

% ---
% Pacotes adicionais, usados no anexo do modelo de folha de identificação
% ---
\usepackage{multicol}
\usepackage{multirow}
% ---
	
% ---
% Pacotes adicionais, usados apenas no âmbito do Modelo Canônico do abnteX2
% ---
\usepackage{lipsum}				% para geração de dummy text
% ---

% ---
% Pacotes de citações
% ---
\usepackage[brazilian,hyperpageref]{backref}	 % Paginas com as citações na bibl
\usepackage[alf]{abntex2cite}	% Citações padrão ABNT

% --- 
% CONFIGURAÇÕES DE PACOTES
% --- 

% ---
% Configurações do pacote backref
% Usado sem a opção hyperpageref de backref
\renewcommand{\backrefpagesname}{Citado na(s) página(s):~}
% Texto padrão antes do número das páginas
\renewcommand{\backref}{}
% Define os textos da citação
\renewcommand*{\backrefalt}[4]{
	\ifcase #1 %
		Nenhuma citação no texto.%
	\or
		Citado na página #2.%
	\else
		Citado #1 vezes nas páginas #2.%
	\fi}%
% ---

% ---
% Informações de dados para CAPA e FOLHA DE ROSTO
% ---
\titulo{Trabalho de Informática}
\autor{Fulano da Silva}
\local{Brasil}
\data{20 de novembro de 2014}
\instituicao{%
  Universidade Federal do Paraná
  \par
  Setor Palotina
  \par
  Engenharia de Aquicultura}
\tipotrabalho{Relatório técnico}
% O preambulo deve conter o tipo do trabalho, o objetivo, 
% o nome da instituição e a área de concentração 
\preambulo{Modelo canônico de Relatório Técnico e/ou Científico em conformidade
com as normas ABNT apresentado à comunidade de usuários \LaTeX.}
% ---

% ---
% Configurações de aparência do PDF final

% alterando o aspecto da cor azul
\definecolor{blue}{RGB}{41,5,195}

% informações do PDF
\makeatletter
\hypersetup{
     	%pagebackref=true,
		pdftitle={\@title}, 
		pdfauthor={\@author},
    	pdfsubject={\imprimirpreambulo},
	    pdfcreator={LaTeX with abnTeX2},
		pdfkeywords={abnt}{latex}{abntex}{abntex2}{relatório técnico}, 
		colorlinks=true,       		% false: boxed links; true: colored links
    	linkcolor=blue,          	% color of internal links
    	citecolor=blue,        		% color of links to bibliography
    	filecolor=magenta,      		% color of file links
		urlcolor=blue,
		bookmarksdepth=4
}
\makeatother
% --- 

% --- 
% Espaçamentos entre linhas e parágrafos 
% --- 

% O tamanho do parágrafo é dado por:
\setlength{\parindent}{1.3cm}

% Controle do espaçamento entre um parágrafo e outro:
\setlength{\parskip}{0.2cm}  % tente também \onelineskip

% ---
% compila o indice
% ---
\makeindex
% ---

% ----
% Início do documento
% ----
\begin{document}

% Retira espaço extra obsoleto entre as frases.
\frenchspacing 

% ----------------------------------------------------------
% ELEMENTOS PRÉ-TEXTUAIS
% ----------------------------------------------------------
% \pretextual

% ---
% Capa
% ---
\imprimircapa
% ---

% ---
% Folha de rosto
% (o * indica que haverá a ficha bibliográfica)
% ---
\imprimirfolhaderosto*
% ---
\tableofcontents

% ---
% Agradecimentos
% ---
% ---

% ---
% ---

% resumo na língua vernácula (obrigatório)


% ---
% inserir lista de ilustrações
% ---
\tableofcontents*s
\listoffigures* %% o * indica que não será incluso no sumário
\cleardoublepage %% Pula página


% ---

% ----------------------------------------------------------
% ELEMENTOS TEXTUAIS  (necessário para incluir número nas páginas)
% ----------------------------------------------------------
\textual




\chapter{Documento de Requisitos} %% NOVO CAPÍTULO (REPARE QUE ELE AUTOMATICAMENTE JÁ COLOCA O NÚMERO DO CAPÍTULO E JÁ ADICIONA NO SUMÁRIO)
\section{Descrição do Sistema}
O sistema para Poíicia Militar consiste em desenvolver um gerenciador de medidas protetivas, expedida pela Justiça. Cada medida tem uma ou mais vitimas e um ou mais Réus, contendo data de emissão, observação e um valor em dias que representa quantos dias esta em vigor essa medida. Com a medida já em mão deve-se cadastradcla colocando esses dados para possível consulta de uma versão previamente digitalizada, assim como criar relatórios online.

O sifdsastema deve ainda emitir diversos tipos de relatórios e consultas, possibilitando um melhor gerenciamento do policiamento ostencivo economisando recursos para melhores aplicações.

\section{Objetivos do documento}

Este documento tem o objetivo de definir os requisitos do sistema medida protetiva. Nele estão contemplados os requisitos funcionais e não-funcionais, casos de uso e regras de negócio do sistema.

\section{Escopo do documento}

Este documento define os requisitos funcionais e não funcionais casos de uso e regras de negócios correlatos para o sistema acima citado. Para elucidação de termos específicos utilizados neste documento, faz-se necessária referência ao Glossário.

\section {Requisitos Funcionais}\label{RF}

\subsection{Login (Usuário/Gerenciador)}\label{RF1}
\begin{enumerate}
	\item O sistema deve possuir 2 niveis de usuários com diferentes privilégios no sistma sendo eles: gerenciador e usuário.\label{login1}
	\item O gerenciador abastecer o sistema podendo consultar e imprimir relatórios.
	\item O usuário apenas consulta e imprimi relatórios.
	
\end{enumerate}

\subsection{Lançamentos das medidas protetivas}\label{RF2} 
\begin{enumerate} 
	\item O sistema deve permitir que apenas o usuário gerenciador realize a inclusão, alteração e remoção de medidas protetivas scaneadas em PDF e anexadas, juntamente do cadastro individual contendo os seguintes atributos: nome, endereço, cidade onde mora, estado, país, telefone, documento de identificação (RG ou CPF para brasileiros e passaporte para estrangeiros), data de nascimento e nome dos pais(“se constar na medida”) de ambas as vitimas e acusados, que constam na médida protetiva.\label{teste}
	
\end{enumerate}

\subsection{Impressão de relatórios e consultas}\label{RF3}
\begin{enumerate}
	\item O sistema deve criar relatórios e gráficos de bairro, cidades, vitimas e acusados com maiores e menores índices de medidas protetivas, cada relatório deve ser individual para cada uma das opções citadas acima, os dois niveis de usuários poderam acessar essa opção.
	\item O sistema deve permitir a consulta de uma medida protetiva online, pelo nome da vítima, nome do acusado ou pelo cpf dos mesmos, tento acesso ao documento original scaneado e armazenado online.
\end{enumerate}


\section{Requisitos não funcionais}

\subsection{Confiabilidade} \label{sec:RFN01}
\begin{enumerate}
	\item O sistema deve ter capacidade para recuperar os dados perdidos da última operação que realizou em caso de falha.
	\item O sistema deve fornecer ferramentas para a realização de backups dos arquivos do sistema.
	\item O sistema deve possuir senhas de acesso e identificação para diferentes tipos de usuários: usuário e gerenciador.
\end{enumerate}


\subsection{Eficiência}\label{sec:RFN02}
\begin{enumerate}
	\item O sistema deve responder a consultas on-line em menos de 5 segundos.
	\item O sistema deve iniciar a impressão de relatórios solicitados dentro de no máximo 20 segundos após sua requisição.
\end{enumerate}
\subsection{Portabilidade}\label{sec:RFN03}
\begin{enumerate}
	\item O sistema deve ser web e adaptavel para dispositivos moveis.
\end{enumerate}
\subsection{Disponibilidade}\label{sec:RFN04}
\begin{enumerate}
	\item O sistema deve estar 24 horas nos 7 dias da semenana online.
\end{enumerate}
%\chapter{Caso de Uso}		
\section{Lista de Casos de Uso}
Alguns dos casos de uso necessários para a implementação do sistema estão listados nesta seção:
\begin{enumerate}
	\item Cadastro de Atores no sistema. \label{Lista:1}
	\item Login de Atores no Sistema.\label{Lista:2}
	\item Cadastro de Medida Protetiva no sistema.\label{Lista:3}
	\item Consulta de Medida Protetiva no sistema.\label{Lista:4}
	\item Editar Medida Protetiva do sistema.\label{Lista:5}
	\item Excluir Medida Protetiva do sistema.\label{Lista:6}
\end{enumerate}
\section{Mapeamento de caso de uso}
Nesta seção os requisitos funcionais (RF) serão mapeados em casos de uso. Os casos de uso que serão implementados, estarão descritos detalhadamente na seção \ref{RF}.

\begin{center}
	\begin{tabular}{ |c|c|c|c| } 
		\hline
		Requisitos funcionais
		RF  & Casos de uso envolvidos
		CU \\
		\hline
		~RF \ref{RF1} & \ref{Lista:1} e \ref{Lista:2} \\ 
		\hline
		~RF \ref{RF2}  e~RF  \ref{RF3} & \ref{Lista:3}, \ref{Lista:4}, \ref{Lista:5} e \ref{Lista:6}   \\ 
		\hline
		
	\end{tabular}
\end{center}

\subsection{Caso de uso cadastro}\label{CU:01}
\begin{table}[H]
	\caption{Caso de Uso de Cadastro}
	\centering
	\includegraphics[width=0.5\textheight]{CU01.png}
\end{table}
\begin{figure}[H]
	\caption{Caso de Uso de Cadastro}
	\centering
	\includegraphics[width=0.5\textheight]{Cadastro.png}
\end{figure}
\subsection{Caso de uso login}\label{CU:02}
\begin{table}[H]
	\caption{Caso de Uso de Login}
	\centering
	\includegraphics[width=0.5\textheight]{CU02.png}
\end{table}
\begin{figure}[H]
	\caption{Caso de Uso de Login}
	\centering
	\includegraphics[width=0.5\textheight]{Login.png}
\end{figure}
\subsection{Caso de uso gerenciador} \label{CU:03}
\begin{table}[H]
	\caption{Caso de Uso de Gerenciador}
	\centering
	\includegraphics[width=0.5\textheight]{CU03.png}
\end{table}
\begin{figure}[H]
	\caption{Caso de Uso de Gerenciador}
	\centering
	\includegraphics[width=0.5\textheight]{Gerenciamento.png}
\end{figure}
\subsection{Caso de uso consultas e relatórios}\label{CU:04}
\begin{table}[H]
	\caption{Caso de Uso de Consultas e Relatórios}
	\centering
	\includegraphics[width=0.5\textheight]{CU04.png}
\end{table}
\begin{figure}[H]
	\caption{Caso de Uso de Consultas e Relatórios}
	\centering
	\includegraphics[width=0.5\textheight]{Consulta-Relatorio.png}
\end{figure}



\section{Lista de Regras de negócio}
\begin{enumerate}
	\item Os gerenciadores só serão aceitos se forem da P2(Setor responsável pela medida protetiva na PM-MS).\label{LISTAN:01}
	\item As atualizações referente a medida protetiva só serão realizadas das 08:00 às 13:30 de Segunda à Sexta feira.\label{LISTAN:02}
\end{enumerate}


\section{Mapeamento de regras de negócio}	
Nesta seção as regras de negócio (RN) serão mapeadas em casos de uso. Os casos de uso que serão implementados, estarão descritos detalhadamente na seção \ref{RF}.

\begin{center}
	\begin{tabular}{ |c|c|c|c| } 
		\hline
		Regras de Negócio
		RN  & Casos de uso envolvidos
		CU \\
		\hline
		~RN \ref{LISTAN:01}  &  CU \ref{CU:03}  \\ 
		\hline
		
	\end{tabular}
\end{center}
\subsection{Setor resposável}
\begin{center}
	
	\begin{tabular}{|l|}
		\hline
		Casos de uso relacionados/Origem das regras de negócio: NÃO HÁ.  
		\\ \hline
		RN \ref{LISTAN:01}: Regras de aprovação de usuário gerenciador.                                                                   \\ \hline
		Descrição: Para ser um gerenciador tem que pertencer Setor P-2. \\ \hline
		Casos de uso relacionados/Origem das regras de negócio: CU \ref{CU:03}.                                                                   \\ \hline
	\end{tabular}
	
\end{center}
\subsection{Abastecimento dos dados}

\begin{center}
	
	\begin{tabular}{|l|}
		
		\hline
		Casos de uso relacionados/Origem das regras de negócio: NÃO HÁ.  
		\\ \hline
		RN \ref{LISTAN:02}: Regras do horário para abastecimento dos dados.                                                                   \\ \hline
		Descrição: Os dados somente serám abastecido  no sistema entre 07h as 14h. \\ \hline
		Casos de uso relacionados/Origem das regras de negócio: NÃO HÁ.                                                                  \\ \hline
	\end{tabular}
	
\end{center}



% ----------------------------------------------------------
% ELEMENTOS PÓS-TEXTUAIS
% ----------------------------------------------------------
\postextual


% ----------------------------------------------------------
% Referências bibliográficas
% ----------------------------------------------------------
\bibliography{abntex2-modelo-references} %% REFERENCIA AO ARQUIVO abntex2-modelo-references.bib

% ----------------------------------------------------------
% Glossário
% ----------------------------------------------------------
%
% Consulte o manual da classe abntex2 para orientações sobre o glossário.
%
\glossary

% ----------------------------------------------------------
% Apêndices
% ----------------------------------------------------------

% ---
% Inicia os apêndices
% ---
\begin{apendicesenv}

% Imprime uma página indicando o início dos apêndices
\partapendices

% ----------------------------------------------------------
\chapter{Quisque libero justo}
% ----------------------------------------------------------

\lipsum[50]

% ----------------------------------------------------------
\chapter{Nullam elementum urna vel imperdiet sodales elit ipsum pharetra ligula
ac pretium ante justo a nulla curabitur tristique arcu eu metus}
% ----------------------------------------------------------
\lipsum[55-57]

\end{apendicesenv}
% ---


% ----------------------------------------------------------
% Anexos
% ----------------------------------------------------------

% ---
% Inicia os anexos
% ---
\begin{anexosenv}

% Imprime uma página indicando o início dos anexos
\partanexos

% ---
\chapter{Morbi ultrices rutrum lorem.}
% ---
\lipsum[30]

% ---
\chapter{Cras non urna sed feugiat cum sociis natoque penatibus et magnis dis
parturient montes nascetur ridiculus mus}
% ---

\lipsum[31]

% ---
\chapter{Fusce facilisis lacinia dui}
% ---

\lipsum[32]

\end{anexosenv}

%---------------------------------------------------------------------
% INDICE REMISSIVO
%---------------------------------------------------------------------

\printindex

%---------------------------------------------------------------------
% Formulário de Identificação (opcional)
%---------------------------------------------------------------------
\chapter*[Formulário de Identificação]{Formulário de Identificação}
\addcontentsline{toc}{chapter}{Exemplo de Formulário de Identificação}
\label{formulado-identificacao}

Exemplo de Formulário de Identificação, compatível com o Anexo A (informativo)
da ABNT NBR 10719:2011. Este formulário não é um anexo. Conforme definido na
norma, ele é o último elemento pós-textual e opcional do relatório.

\bigskip

\begin{tabular}{|p{9cm}|p{5cm}|} %% EXEMPLO DE TABELA MAIS COMPLEXA
\hline
\multicolumn{2}{|c|}{\textbf{\large Dados do Relatório Técnico e/ou científico}}\\
\hline
\multirow{4}{10cm}[24pt]{Título e subtítulo}& Classificação de segurança\\
                   & \\
                   \cline{2-2}
                   & No.\\
                   & \\
				
\hline
Tipo de relatório & Data\\
\hline
Título do projeto/programa/plano & No.\\
\hline
\multicolumn{2}{|l|}{Autor(es)} \\
\hline
\multicolumn{2}{|l|}{Instituição executora e endereço completo} \\
\hline
\multicolumn{2}{|l|}{Instituição patrocinadora e endereço completo} \\
\hline
\multicolumn{2}{|l|}{Resumo}\\[3cm]
\hline
\multicolumn{2}{|l|}{Palavras-chave/descritores}\\
\hline
\multicolumn{2}{|l|}{
Edição \hfill No. de páginas \hfill No. do volume \hfill Nº de classificação \phantom{XXXX}} \\
\hline
\multicolumn{2}{|l|}{
ISSN \hfill \hfill Tiragem \hfill Preço \phantom{XXXXXXXX}} \\
\hline
\multicolumn{2}{|l|}{Distribuidor} \\
\hline
\multicolumn{2}{|l|}{Observações/notas}\\[3cm]
\hline
\end{tabular}

\end{document}
